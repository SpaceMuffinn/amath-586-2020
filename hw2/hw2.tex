\documentclass[10pt]{amsart}
\usepackage[margin=1.5in]{geometry}
\usepackage{amssymb,amsmath,enumitem}

\DeclareMathOperator{\D}{d}
\DeclareMathOperator{\E}{e}
\newcommand{\half}{\frac{1}{2}}
\newcommand\unp{U^{n+1}}
\newcommand\unm{U^{n-1}}
\newcommand{\bigo}{{\mathrm O}}

\begin{document}

%\topmargin -1.0in
%\textheight 10.5in
\pagestyle{empty}

\newcommand{\mline}{\vspace{.2in}\hrule\vspace{.2in}}


\title{\bf { AMATH 586 Spring 2020 \\ Homework 2 ---
Due April 24 on GitHub by 11pm} }
\maketitle
\centerline{Be sure to do a {\tt git pull} to update your local version of the {\tt amath-586-2020} repository.}

\mline
\begin{enumerate}[label={\bf Problem~{\arabic*}:}]
  \item Consider

$$ v'''(t) + v'(t) v(t) - \frac{\beta_1 + \beta_2 + \beta_3}{3} v'(t) =0, $$

where $\beta_1 < \beta_2 < \beta_3$.  It follows that
$$
v(t) = \beta_2 + (\beta_3 - \beta_2) \mathrm{cn}^2\left( \frac{\sqrt{ \beta_3 - \beta_1}}{12} t, \sqrt{\frac{\beta_3 - \beta_2}{\beta_3 - \beta_1}} \right)
$$
is a solution where $\mathrm{cn}(x,k)$ is the Jacobi cosine function and $k$ is the elliptic modulus.  Some notations use $\mathrm{cn}(x,m)$ where $m = k^2$.  The corresponding initial conditions are
$$
v(0) = \beta_3, \\
v'(0) = 0,\\
v''(0) = -\frac{(\beta_3 - \beta_1)(\beta_3-\beta_2)}{6}.$$
Derive a third-order Runge-Kutta method and verify the order of accuracy on this problem using the methodology in Lecture 7 --- produce a plot and and a table.
\mline
\item  Which of the following Linear Multistep Methods are convergent?  For 
the ones that are not, are they inconsistent, or not zero-stable, or both?
 \begin{enumerate}
 \item $U^{n+3} = U^{n+1} + 2kf(U^n)$,
 \item $U^{n+2} = \half U^{n+1} + \half U^{n} + 2kf(U^{n+1})$,
 \item $\unp = U^n$, 
 \item $U^{n+4} = U^{n} + \frac 4 3 k(f(U^{n+3})+f(U^{n+2})+f(U^{n+1}))$,
 \item $U^{n+3} = -U^{n+2} + U^{n+1} +U^{n}+2k(f(U^{n+2})+f(U^{n+1}))$.
 \end{enumerate}

 \mline
 \item For the one-step method (6.17) show that the Lipschitz constant is $L' = L + \frac{k}{2} L^2$ where $L$ is the Lipschitz constant for $f$.


% uncomment the next two lines if you want to insert solution...
%\vskip 1cm
%{\bf Solution:}

% insert your solution here!

\mline 
\item  The Fibonacci numbers

\begin{enumerate}
\item Determine the general solution to the linear difference equation
$U^{n+2} = U^{n+1} + U^n$.

\item Determine the solution to this difference equation with the starting
values $U^0=1$, $U^1=1$.  Use this to determine $U^{30}$.
(Note, these are the {\em Fibonacci numbers}, which of course should all be
integers.)

\item Show that for large $n$ the ratio of successive Fibonacci numbers
$U^n/U^{n-1}$ approaches the ``golden ratio'' $\phi \approx 1.618034$.
\end{enumerate} 

% uncomment the next two lines if you want to insert solution...
%\vskip 1cm
%{\bf Solution:}

% insert your solution here!

%--------------------------------------------------------------------------
\mline
\item 
Any $r$-stage Runge-Kutta method applied to $u'=\lambda u$ will give an
expression of the form
\[
U^{n+1} = R(z)U^n
\]
where $z=\lambda k$ and $R(z)$ is a rational function, a ratio of
polynomials in $z$ each having degree at most $r$.  For an explicit method
$R(z)$ will simply be a polynomial of degree $r$ and for an implicit method
it will be a more general rational function.

Since $u(t_{n+1}) = e^z u(t_n)$ for this problem, we expect that a $p$th
order accurate method will give a function $R(z)$ satisfying
\[
\qquad  R(z) = e^z + \bigo(z^{p+1}) \quad\text{as}~z \to 0.
\]
This indicates that the one-step error is $\bigo(z^{p+1})$ on this problem,
as expected for a $p$th order accurate method.

The explicit
Runge-Kutta method of Example 5.13 is fourth order accurate in general,
so in particular it should exhibit this accuracy when applied to 
$u'(t) = \lambda u(t)$.  Show that in fact when applied to this
problem the method becomes $U^{n+1} = R(z)U^n$ where $R(z)$ is 
a polynomial of degree 4, and that this polynomial agrees with the Taylor
expansion of $e^z$ through $O(z^4)$ terms.

We will see that this function $R(z)$ is also important in the study of 
absolute stability of a one-step method.

% uncomment the next two lines if you want to insert solution...
%\vskip 1cm
%{\bf Solution:}

% insert your solution here!

%--------------------------------------------------------------------------

\mline
\item

Determine the function $R(z)$ described in the previous exercise for the
TR-BDF2 method given in (5.37).  Note that this can be simplified to the
form (8.6), which is given only for the autonomous case but that suffices
for $u'(t) = \lambda u(t)$.  (You might want to convince yourself these are 
the same method).

Confirm that $R(z)$ agrees with $e^z$ to the expected order.

Note that for this implicit method $R(z)$ will be a rational function, so you
will have to expand the denominator in a Taylor series, or use the Neumann
series 
\[
1/(1-\epsilon) = 1 + \epsilon + \epsilon^2 + \epsilon^3 + \cdots.
\]



  
  \end{enumerate}

\end{document}

%%% Local Variables:
%%% mode: latex
%%% TeX-master: t
%%% End:
